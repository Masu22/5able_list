\documentclass[12pt]{article}
\usepackage{amsmath, amssymb, amsthm}
\usepackage{hyperref}

\title{数学の分野ごとのまとめ}
\author{あなたの名前}
\date{\today}

% 定義環境と定理環境の設定
\newtheorem{definition}{定義}[section]  % セクションごとに番号が振られる
\newtheorem{theorem}{定理}[section]     % セクションごとに番号が振られる
\newtheorem{lemma}{補題}[section]       % セクションごとに番号が振られる
\newtheorem{corollary}{系}[section]     % セクションごとに番号が振られる
\newtheorem{problem}{問題}[section]
\newtheorem{solution}{解法}[section]

\begin{document}

\maketitle

\tableofcontents  % 目次の自動生成

\section{前提知識 (Preliminary Knowledge)}
\subsection{定義 (Definitions)}
% ここに定義を追加します
\begin{definition}
    数列 $a_n$ が収束するとは、$\lim_{n \to \infty} a_n = L$ となることです。
\end{definition}

\subsection{定理 (Theorems)}
% ここに定理を追加します
\begin{theorem}
    $\lim_{n \to \infty} a_n = L$ が成り立つならば、数列 $a_n$ は収束すると言います。
\end{theorem}

\section{定理と証明 (Theorems and Proofs)}
\begin{theorem}[例: 関数の極限]
    $f(x)$ が $x_0$ で連続ならば、$\lim_{x \to x_0} f(x) = f(x_0)$。
\end{theorem}

\begin{proof}
    証明の詳細。
\end{proof}

\section{例 (Examples)}
\subsection{例1: 数列の収束}
数列 $a_n = \frac{1}{n}$ は $0$ に収束する。

\subsection{例2: 関数の極限}
関数 $f(x) = x^2$ は $x_0 = 2$ で連続であり、$\lim_{x \to 2} f(x) = 4$。

\section{補題 (Lemmas)}
\begin{lemma}
    任意の実数列 $a_n$ が上に有界ならば、部分列の中で収束するものが存在する。
\end{lemma}

\begin{proof}
    補題の証明。
\end{proof}

\section{問題 (Problems)}
\begin{problem}
    数列 $a_n = \frac{2n+1}{n^2 + 1}$ の極限を求めよ。
\end{problem}

\begin{solution}
    解答の詳細。
\end{solution}

\section{文献 (References)}
\begin{thebibliography}{99}
    \bibitem{example1} 書籍名, 著者, 出版年.
    \bibitem{example2} 論文名, 著者, 雑誌名, 年.
\end{thebibliography}

\appendix
\section{付録 (Appendix)}
ここには計算の詳細や補足説明を追加します。

\end{document}

