{\LARGE \bf{Category Theory}}
\section{Introduction to Category Theory}
1.1 Historical context and motivation for category theory\
1.2 Basic definitions and notation\
1.3 Examples of categories from various mathematical fields\
1.4 Category theory as a unifying language for mathematics\
\section{Categories and Morphisms}
2.1 Formal definition of a category\
2.2 Properties and types of morphisms\
2.3 Composition and identity morphisms\
2.4 Commutative diagrams and their interpretation\
\section{Isomorphisms and Initial and Terminal Objects}
\section{Isomorphisms, Initial \& Terminal Objects}
3.1 Isomorphisms and their properties\
3.2 Initial and terminal objects\
3.3 Uniqueness up to unique isomorphism\
3.4 Examples in concrete categories\
\section{Functors and Functor Categories}
4.1 Definition and types of functors\
4.2 Functor composition and identity functors\
4.3 Faithful, full, and essentially surjective functors\
4.4 Functor categories and natural transformations\
\section{Natural Transformations}
5.1 Definition and examples of natural transformations\
5.2 Vertical and horizontal composition of natural transformations\
5.3 Natural isomorphisms and equivalences\
5.4 The category of functors and natural transformations\
\section{Equivalence of Categories}
6.1 Definition of category equivalence\
6.2 Essentially surjective and fully faithful functors\
6.3 Skeletal categories and skeletons\
6.4 Examples and applications of category equivalence\
\section{Universal Properties and Limits}
7.1 Universal properties and universal arrows\
7.2 Definition and examples of limits\
7.3 Products, equalizers, and pullbacks\
7.4 Completeness and preservation of limits\
\section{Colimits and Kan Extensions}
8.1 Definition and examples of colimits\
8.2 Coproducts, coequalizers, and pushouts\
8.3 Kan extensions: left and right\
8.4 Applications of Kan extensions\
\section{Adjunctions}
9.1 Definition and examples of adjoint functors\
9.2 Unit and counit of an adjunction\
9.3 Adjunctions and universal properties\
9.4 Adjoint functor theorems\
\section{Monads and Algebras}
10.1 Definition and examples of monads\
10.2 Algebras for a monad\
10.3 The Eilenberg-Moore category\
10.4 Kleisli category and free algebras\
\section{Monoidal Categories and Braiding}
11.1 Definition and examples of monoidal categories\
11.2 Symmetric monoidal categories\
11.3 Braided monoidal categories\
11.4 Coherence theorems\
\section{Duality and Opposite Categories}
12.1 Opposite categories and duality principle\
12.2 Contravariant functors\
12.3 Dual notions: limits and colimits, initial and terminal objects\
12.4 Applications of duality in various mathematical contexts\
\section{The Yoneda Lemma and Presheaves}
13.1 Representable functors and the Yoneda embedding\
13.2 The Yoneda lemma and its consequences\
13.3 Presheaves and the category of presheaves\
13.4 Applications of the Yoneda lemma\
\section{Topoi and Geometric Morphisms}
14.1 Definition and examples of topoi\
14.2 Subobject classifier and power objects\
14.3 Geometric morphisms between topoi\
14.4 Logic in topoi and sheaf theory\
\section{Applications to Algebra and Topology}
15.1 Galois connections and Galois theory\
15.2 Abelian categories and homological algebra\
15.3 Homotopy theory and model categories\
15.4 Topos theory in algebraic geometry\
