{\LARGE \bf{Mathematical Crystallography}}
\section{Crystallography: Intro to Symmetry}
\section{Introduction to Crystallography and Symmetry}
1.1 Historical development of crystallography\
1.2 Basic concepts of symmetry and crystal structures\
1.3 Crystallographic notation and terminology\
1.4 Introduction to crystal systems and lattices\
\section{Fundamentals of Group Theory}
2.1 Basic group theory concepts and definitions\
2.2 Symmetry operations and group representations\
2.3 Subgroups, cosets, and isomorphisms\
2.4 Applications of group theory in crystallography\
\section{Point Groups \& Crystal Classes}
\section{Point Groups and Crystal Classes}
3.1 Definition and classification of point groups\
3.2 Stereographic projections and crystal classes\
3.3 Symmetry elements and their combinations\
3.4 Point group determination and analysis\
\section{Translation Symmetry and Bravais Lattices}
\section{Translation Symmetry and Lattices}
4.1 Lattice types and Bravais lattices\
4.2 Unit cells and crystal systems\
4.3 Lattice planes and Miller indices\
4.4 Crystallographic directions and zone axes\
\section{Space Groups and Their Notation}
\section{Space Groups: Notation and Classification}
5.1 Space group symmetry and operations\
5.2 International Tables for Crystallography notation\
5.3 Systematic absences and extinction rules\
5.4 Space group determination and analysis\
\section{Symmetry Operations \& Generators}
\section{Symmetry Operations and Generators}
6.1 Symmetry operation matrices and algebra\
6.2 Generators and minimal sets of symmetry operations\
6.3 Wyckoff positions and site symmetry\
6.4 Symmetry constraints on crystal structures\
\section{Diffraction Theory Fundamentals}
\section{Introduction to Diffraction Theory}
7.1 Wave-particle duality and Bragg's law\
7.2 Scattering by atoms and crystals\
7.3 Ewald sphere and reciprocal lattice\
7.4 Intensity of diffracted beams and structure factors\
\section{Structure Factors \& Fourier Transforms}
\section{Structure Factors and Fourier Transforms}
8.1 Structure factor calculation and properties\
8.2 Fourier transforms in crystallography\
8.3 Patterson function and heavy atom methods\
8.4 Direct methods for phase determination\
\section{Reciprocal Space \& Brillouin Zones}
\section{Reciprocal Space and Brillouin Zones}
9.1 Reciprocal lattice construction and properties\
9.2 Brillouin zones and their significance\
9.3 Reciprocal space mapping techniques\
9.4 Applications in solid-state physics and materials science\
\section{X-ray Diffraction Techniques and Applications}
\section{X-ray Diffraction: Techniques \& Applications}
10.1 X-ray sources and instrumentation\
10.2 Single crystal and powder diffraction methods\
10.3 Data collection strategies and processing\
10.4 Qualitative and quantitative phase analysis\
\section{Electron and Neutron Diffraction Methods}
\section{Electron and Neutron Diffraction}
11.1 Electron diffraction theory and techniques\
11.2 Neutron diffraction principles and applications\
11.3 Comparison of X-ray, electron, and neutron diffraction\
11.4 Specialized diffraction methods (e.g., LEED, RHEED)\
\section{Crystal Structure Determination \& Refinement}
\section{Crystal Structure Determination and Refinement}
12.1 Structure solution strategies\
12.2 Refinement methods and least-squares analysis\
12.3 Accuracy, precision, and error analysis\
12.4 Advanced refinement techniques (e.g., disorder, twinning)\
\section{Tensor Properties of Crystals}
13.1 Introduction to tensor notation and properties\
13.2 Symmetry constraints on tensor properties\
13.3 Elastic, piezoelectric, and dielectric tensors\
13.4 Optical properties and indicatrix\
\section{Physical Properties and Crystal Symmetry}
\section{Physical Properties and Symmetry Relations}
14.1 Neumann's principle and property tensors\
14.2 Thermal expansion and thermoelectric effects\
14.3 Magnetic properties and symmetry\
14.4 Ferroic materials and phase transitions\
\section{Quasicrystals and Aperiodic Structures}
15.1 Discovery and characteristics of quasicrystals\
15.2 Penrose tilings and higher-dimensional approaches\
15.3 Diffraction patterns of quasicrystals\
15.4 Incommensurate and composite structures\
\section{Modulated Structures \& Superspace Groups}
\section{Modulated Structures and Superspace Groups}
16.1 Types of modulated structures\
16.2 Superspace description and symmetry\
16.3 Diffraction patterns of modulated structures\
16.4 Structure solution and refinement in superspace\
\section{Computational Methods in Crystal Analysis}
\section{Computational Methods in Crystallography}
17.1 Crystallographic software packages and databases\
17.2 Structure visualization and analysis tools\
17.3 Ab initio structure prediction methods\
17.4 Machine learning applications in crystallography\
\section{Advanced Topics and Current Research Trends}
\section{Advanced Topics in Mathematical Crystallography}
18.1 Time-resolved crystallography\
18.2 High-pressure and extreme condition crystallography\
18.3 Nanocrystallography and electron crystallography\
18.4 Biomolecular crystallography and drug design\
