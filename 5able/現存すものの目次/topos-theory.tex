{\LARGE \bf{Topos Theory}}
\section{Introduction to Category Theory}
1.1 Basic definitions and examples of categories\
1.2 Morphisms, isomorphisms, and functors\
1.3 Duality and opposite categories\
1.4 Special objects: initial, terminal, and zero objects\
\section{Functors and Natural Transformations}
2.1 Types of functors: covariant, contravariant, and bifunctors\
2.2 Natural transformations and their properties\
2.3 Functor categories and the Yoneda lemma\
\section{Limits and Colimits}
3.1 Universal properties and representable functors\
3.2 Limits, colimits, and their basic properties\
3.3 Special cases: products, coproducts, equalizers, and coequalizers\
3.4 Completeness and cocompleteness of categories\
\section{Adjoint Functors}
4.1 Definition and properties of adjoint functors\
4.2 Unit and counit of adjunction\
4.3 Examples and applications of adjunctions\
\section{Cartesian Closed Categories}
5.1 Definition and properties of cartesian closed categories\
5.2 Exponential objects and evaluation morphisms\
5.3 Examples of cartesian closed categories\
\section{Subobject Classifiers and Topoi}
6.1 Subobjects and characteristic functions\
6.2 Definition and properties of subobject classifiers\
6.3 Elementary topoi and their basic properties\
\section{Elementary Topoi and Examples}
7.1 Axioms and properties of elementary topoi\
7.2 Set-based topoi and finite topoi\
7.3 Presheaf topoi and functor categories\
\section{Sheaves and Sheafification}
8.1 Presheaves and their properties\
8.2 Definition and examples of sheaves\
8.3 Sheafification and associated sheaf functor\
\section{Grothendieck Topoi}
9.1 Sites and Grothendieck topologies\
9.2 Definition and properties of Grothendieck topoi\
9.3 Comparison with elementary topoi\
\section{Geometric Morphisms and Classifying Topoi}
10.1 Definition and properties of geometric morphisms\
10.2 Classifying topoi and universal properties\
10.3 Applications to algebraic geometry and model theory\
\section{Internal Logic and Mitchell-Bénabou Language}
\section{Internal Logic and Mitchell–Bénabou Language}
11.1 Internal language of a topos\
11.2 Mitchell-Bénabou language and its semantics\
11.3 Kripke-Joyal semantics and forcing\
\section{Topos Theory: Algebraic and Geometric Views}
\section{Topos-Theoretic Approaches to Algebra and Geometry}
12.1 Algebraic theories in topoi\
12.2 Synthetic differential geometry\
12.3 Topological and smooth topoi\
\section{Topos Theory and Foundations of Mathematics}
13.1 Intuitionistic logic and constructive mathematics\
13.2 Set theory in topoi\
13.3 Independence results and alternative foundations\
\section{Applications and Further Developments in Topos Theory}
\section{Topos Theory: Advanced Applications}
14.1 Topoi in computer science and logic\
14.2 Cohomology theories and topos theory\
14.3 Higher-dimensional and ∞-topoi\
