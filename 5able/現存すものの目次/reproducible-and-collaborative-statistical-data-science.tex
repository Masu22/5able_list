{\LARGE \bf{Collaborative Data Science}}
\section{Reproducible Research Fundamentals}
1.1 Principles of reproducibility\
1.2 Replication crisis in science\
1.3 Research transparency\
1.4 Computational reproducibility\
1.5 Open data and open methods\
1.6 Reproducible workflows\
1.7 Reproducibility tools and platforms\
\section{Version Control in Data Science}
2.1 Git fundamentals\
2.2 GitHub and GitLab\
2.3 Branching and merging\
2.4 Collaborative development with pull requests\
2.5 Conflict resolution\
2.6 Version control best practices\
2.7 Git for data science projects\
\section{Data Management for Statistical Research}
3.1 Data storage formats\
3.2 File naming conventions\
3.3 Directory structure\
3.4 Metadata standards\
3.5 Data cleaning and preprocessing\
3.6 Data versioning\
3.7 Data sharing and archiving\
\section{Statistical Programming Languages}
4.1 R programming\
4.2 Python for data science\
4.3 SQL for data manipulation\
4.4 Julia for scientific computing\
4.5 SAS and SPSS\
4.6 Language interoperability\
4.7 Choosing the right language for a project\
\section{Collaborative Coding Best Practices}
5.1 Code review processes\
5.2 Pair programming\
5.3 Agile methodologies in data science\
5.4 Collaborative platforms and tools\
5.5 Code style guides and linting\
5.6 Testing and continuous integration\
5.7 Managing dependencies and environments\
\section{Data Visualization Essentials}
6.1 Principles of effective data visualization\
6.2 Static visualizations\
6.3 Interactive visualizations\
6.4 Geospatial visualizations\
6.5 Network and graph visualizations\
6.6 Time series visualizations\
6.7 Dashboard creation\
\section{Statistical Analysis Techniques}
7.1 Descriptive statistics\
7.2 Inferential statistics\
7.3 Regression analysis\
7.4 Analysis of variance (ANOVA)\
7.5 Time series analysis\
7.6 Bayesian statistics\
7.7 Multivariate analysis\
\section{Machine Learning Fundamentals}
8.1 Supervised learning\
8.2 Unsupervised learning\
8.3 Deep learning\
8.4 Ensemble methods\
8.5 Feature selection and engineering\
8.6 Model evaluation and validation\
8.7 Hyperparameter tuning\
\section{Project Workflow in Data Science}
9.1 Project planning and scoping\
9.2 Task management and prioritization\
9.3 Reproducible analysis pipelines\
9.4 Containerization with Docker\
9.5 Workflow automation tools\
9.6 Resource management\
9.7 Project delivery and deployment\
\section{Documentation and Literate Programming}
10.1 Code documentation best practices\
10.2 Markdown and reStructuredText\
10.3 Jupyter notebooks\
10.4 R Markdown\
10.5 Automated documentation tools\
10.6 API documentation\
10.7 Writing reproducible reports\
\section{Open Science: Principles and Practices}
11.1 Open access publishing\
11.2 Preregistration of studies\
11.3 Data sharing platforms\
11.4 Open source software development\
11.5 Citizen science\
11.6 Research ethics in open science\
11.7 Impact and metrics of open science\
\section{Reproducibility Across Domains}
12.1 Reproducibility in biomedical research\
12.2 Reproducibility in social sciences\
12.3 Reproducibility in environmental sciences\
12.4 Reproducibility in physics and astronomy\
12.5 Reproducibility in economics\
12.6 Reproducibility in computer science\
12.7 Cross-domain reproducibility challenges\
