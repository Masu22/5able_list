{\LARGE \bf{Sampling Surveys}}
\section{Introduction to Sampling Surveys}
1.1 Fundamentals of sampling theory\
1.2 Historical development of sampling methods\
1.3 Types of sampling designs\
1.4 Applications of sampling surveys in various fields\
\section{Sampling Design and Techniques}
2.1 Probability and non-probability sampling\
2.2 Sampling frames and their importance\
2.3 Sampling units and sampling errors\
2.4 Comparison of sampling techniques\
\section{Simple Random Sampling}
3.1 Principles of simple random sampling\
3.2 Methods for selecting simple random samples\
3.3 Estimation and inference in simple random sampling\
3.4 Advantages and limitations of simple random sampling\
\section{Stratified Sampling}
4.1 Concept and rationale of stratified sampling\
4.2 Methods of stratification\
4.3 Allocation of sample sizes to strata\
4.4 Analysis and estimation in stratified sampling\
\section{Cluster Sampling}
5.1 Principles of cluster sampling\
5.2 One-stage and two-stage cluster sampling\
5.3 Estimation in cluster sampling\
5.4 Efficiency considerations in cluster sampling\
\section{Multistage Sampling}
6.1 Concept and design of multistage sampling\
6.2 Probability proportional to size (PPS) sampling\
6.3 Estimation procedures in multistage sampling\
6.4 Applications of multistage sampling\
\section{Survey Design and Questionnaire Construction}
\section{Survey Design and Questionnaire Development}
7.1 Principles of effective survey design\
7.2 Types of questions and response formats\
7.3 Questionnaire layout and organization\
7.4 Pilot testing and questionnaire refinement\
\section{Sampling and Nonsampling Errors}
8.1 Sources of sampling errors\
8.2 Types of nonsampling errors\
8.3 Measurement and control of errors\
8.4 Impact of errors on survey results\
\section{Sample Size Determination}
9.1 Factors affecting sample size\
9.2 Sample size calculation methods\
9.3 Power analysis in sampling\
9.4 Optimal allocation of resources\
\section{Data Collection Methods}
10.1 Face-to-face interviews\
10.2 Telephone and online surveys\
10.3 Mail surveys and self-administered questionnaires\
10.4 Mixed-mode data collection strategies\
\section{Statistical Analysis of Survey Data}
11.1 Descriptive statistics for survey data\
11.2 Inferential statistics and hypothesis testing\
11.3 Regression analysis in survey research\
11.4 Multivariate analysis techniques\
\section{Nonresponse and Missing Data}
12.1 Types and causes of nonresponse\
12.2 Nonresponse bias and its effects\
12.3 Techniques for handling missing data\
12.4 Imputation methods and their applications\
\section{Bias Correction Techniques}
13.1 Weighting adjustments\
13.2 Post-stratification and calibration\
13.3 Propensity score methods\
13.4 Bias analysis and sensitivity testing\
\section{Practical Applications of Sampling Surveys}
\section{Sampling Surveys: Practical Applications}
14.1 Market research and opinion polls\
14.2 Social and demographic surveys\
14.3 Health and medical research applications\
14.4 Environmental and ecological sampling\
\section{Ethical Considerations in Sampling Surveys}
\section{Ethical Considerations in Survey Sampling}
15.1 Informed consent and respondent rights\
15.2 Confidentiality and data protection\
15.3 Ethical issues in sensitive topic research\
15.4 Professional codes of conduct in survey research\
