{\LARGE \bf{Knot Theory}}
\section{Introduction to Knot Theory and Basic Definitions}
\section{Knot Theory Basics: Intro \& Definitions}
1.1 History and development of knot theory\
1.2 Basic definitions: knots, links, and embeddings\
1.3 Ambient isotopy and knot equivalence\
1.4 Orientation and chirality of knots\
\section{Knot Diagrams and Reidemeister Moves}
2.1 Knot diagrams and projections\
2.2 Reidemeister moves and their significance\
2.3 Planar isotopy and regular isotopy\
\section{Knot Equivalence and Knot Invariants}
3.1 Concept of knot invariants\
3.2 Crossing number and bridge number\
3.3 Tricolorability and other coloring invariants\
3.4 Unknotting number and slice genus\
\section{The Fundamental Group and the Knot Group}
4.1 Introduction to the fundamental group\
4.2 Knot group and Wirtinger presentation\
4.3 Applications of knot groups in distinguishing knots\
\section{Seifert Surfaces and Genus}
5.1 Seifert surfaces and their construction\
5.2 Genus of a knot and its properties\
5.3 Seifert matrices and their applications\
\section{The Alexander Polynomial}
6.1 Definition and properties of the Alexander polynomial\
6.2 Computation techniques for the Alexander polynomial\
6.3 Applications and limitations of the Alexander polynomial\
\section{The Jones Polynomial and Kauffman Bracket}
7.1 Introduction to the Jones polynomial\
7.2 The Kauffman bracket and its relation to the Jones polynomial\
7.3 Properties and applications of the Jones polynomial\
\section{Other Polynomial Invariants: HOMFLY and Kauffman Polynomials}
\section{Polynomial Invariants: HOMFLY and Kauffman}
8.1 The HOMFLY polynomial: definition and properties\
8.2 The Kauffman polynomial: definition and properties\
8.3 Comparisons and relationships between polynomial invariants\
\section{Knot Tabulation and Classification}
9.1 Historical development of knot tables\
9.2 Classification of knots up to certain crossing numbers\
9.3 Computational methods in knot tabulation\
\section{Braids and the Braid Group}
10.1 Introduction to braids and the braid group\
10.2 Artin's braid theory and Markov's theorem\
10.3 Relationships between braids and knots\
\section{Links and Link Invariants}
11.1 Multi-component links and their properties\
11.2 Linking number and other numerical link invariants\
11.3 Milnor invariants and higher-order linking\
\section{Khovanov Homology and Categorification}
12.1 Introduction to homology theories in knot theory\
12.2 Khovanov homology: definition and construction\
12.3 Applications and recent developments in categorification\
\section{Three-Manifolds and Knot Complements}
\section{Three–Manifolds and Knot Complements}
13.1 Introduction to 3-manifolds and their relationship to knots\
13.2 Knot complements and their topological properties\
13.3 Dehn surgery and its applications in knot theory\
\section{Applications of Knot Theory in Biology and Chemistry}
\section{Knot Theory in Biology and Chemistry}
14.1 DNA topology and knotting in molecular biology\
14.2 Knots in proteins and other biomolecules\
14.3 Chemical topology and molecular knots\
\section{Applications of Knot Theory in Physics and Quantum Field Theory}
\section{Knot Theory in Physics and Quantum Fields}
15.1 Knots in statistical mechanics and polymer physics\
15.2 Topological quantum field theories and knot invariants\
15.3 Knots in string theory and other areas of theoretical physics\
