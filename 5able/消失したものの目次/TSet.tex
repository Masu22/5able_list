{\LARGE \bf{Introduction to The Theory of Sets}}
\section{Set Theory: Foundations and Key Concepts}
1.1 Definition and notation of sets
1.2 Set membership and subset relations
1.3 Set-builder notation and Russell's Paradox
1.4 Power sets and universal sets
\section{Set Operations and Venn Diagrams}
2.1 Union, intersection, and complement operations
2.2 Venn diagrams and set visualization
2.3 De Morgan's laws and algebraic properties of sets
\section{Relations and Functions in Set Theory}
3.1 Cartesian products and ordered pairs
3.2 Binary relations and their properties
3.3 Functions as special relations
3.4 Injective, surjective, and bijective functions
\section{Equivalence Relations \& Partial Orders}
4.1 Equivalence relations and partition of sets
4.2 Partial orders and their properties
4.3 Total orders and well-orders
\section{Axioms of Set Theory: ZFC System}
5.1 Zermelo-Fraenkel axioms
5.2 The Axiom of Choice and its equivalents
5.3 Consistency and independence of axioms
\section{Finite vs. Infinite Sets in Set Theory}
6.1 Finite sets and their properties
6.2 Infinite sets and Dedekind-infinite sets
6.3 Comparing set sizes and Cantor's theorem
\section{Countable vs Uncountable Sets}
7.1 Countable sets and their properties
7.2 Uncountable sets and Cantor's diagonalization
7.3 The continuum and its properties
\section{Cardinal Numbers and Arithmetic}
8.1 Definition and properties of cardinal numbers
8.2 Cardinal arithmetic operations
8.3 The Continuum Hypothesis and Generalized Continuum Hypothesis
\section{Ordinal Numbers \& Transfinite Induction}
9.1 Ordinal numbers and their arithmetic
9.2 Transfinite induction and recursion
9.3 Well-ordering theorem and Zorn's lemma
\section{Continuum Hypothesis: Implications \& Impact}
10.1 Historical context and importance of the Continuum Hypothesis
10.2 Gödel's constructible universe and consistency of CH
10.3 Forcing and the independence of CH
\section{Set Paradoxes: Challenges and Solutions}
11.1 Russell's Paradox and the development of axiomatic set theory
11.2 The Burali-Forti Paradox and ordinal numbers
11.3 Cantor's Paradox and limitations of naive set theory
\section{Set Theory in Math and Computer Science}
12.1 Set theory in topology and analysis
12.2 Set-theoretic foundations of computer science
12.3 Set theory in logic and model theory
12.4 Contemporary research directions in set theory
