{\LARGE \bf{Fractal Geometry}}
\section{Introduction to Fractals}
1.1 Definition and properties of fractals
1.2 Self-similarity and scale invariance
1.3 History and development of fractal geometry
1.4 Applications of fractals in various fields
\section{Dimensions and Measures}
2.1 Euclidean dimensions and their limitations
2.2 Hausdorff dimension and box-counting dimension
2.3 Fractal dimensions and their calculation
2.4 Measures of fractal sets and their properties
\section{Iterated Function Systems (IFS)}
3.1 Definition and properties of IFS
3.2 Contractive mappings and fixed points
3.3 Constructing fractals using IFS (e.g., Sierpinski triangle, Cantor set)
3.4 Chaos game and random iteration algorithm
\section{Julia Sets and the Mandelbrot Set}
4.1 Complex dynamics and iteration of complex functions
4.2 Definition and properties of Julia sets
4.3 The Mandelbrot set and its relationship to Julia sets
4.4 Exploring the Mandelbrot set and its structures
\section{L–Systems and Fractal Plants}
5.1 Introduction to L-systems and their properties
5.2 Deterministic and stochastic L-systems
5.3 Generating fractal plants and trees using L-systems
5.4 Applications of L-systems in computer graphics and modeling
\section{Fractal Interpolation and Curves}
6.1 Fractal interpolation functions and their construction
6.2 Self-affine and self-similar curves
6.3 Space-filling curves (e.g., Hilbert curve, Peano curve)
6.4 Applications of fractal curves in signal processing and data compression
\section{Random Fractals and Brownian Motion}
7.1 Introduction to random fractals and their properties
7.2 Fractional Brownian motion and its characteristics
7.3 Generating random fractals using midpoint displacement and other methods
7.4 Applications of random fractals in modeling natural phenomena
\section{Multifractals and Measures}
8.1 Definition and properties of multifractals
8.2 Multifractal measures and their construction
8.3 Multifractal spectrum and its calculation
8.4 Applications of multifractals in turbulence and financial markets
\section{Fractal Image Compression}
9.1 Principles of fractal image compression
9.2 Partitioned iterated function systems (PIFS)
9.3 Encoding and decoding algorithms for fractal image compression
9.4 Advantages and limitations of fractal image compression
\section{Fractals in Nature and Art}
10.1 Fractal patterns in natural objects and phenomena
10.2 Fractal analysis of biological structures and systems
10.3 Fractals in art, music, and architecture
10.4 Aesthetic and philosophical aspects of fractals
\section{Computational Methods and Software}
11.1 Numerical methods for generating and analyzing fractals
11.2 Fractal software packages and libraries
11.3 Programming fractals using various languages (e.g., Python, MATLAB)
11.4 Visualization techniques for fractal sets and measures
\section{Advanced Topics and Research}
12.1 Current research trends in fractal geometry
12.2 Fractals in higher dimensions and abstract spaces
12.3 Connections between fractals and other mathematical fields
12.4 Open problems and future directions in fractal geometry
