{\LARGE \bf{K theory}}

\section{Introduction to Vector Bundles and K–Theory}
1.1 Basic concepts of vector bundles\\
1.2 Continuous and smooth vector bundles\\
1.3 Operations on vector bundles\\
1.4 Introduction to K-Theory and its motivations
\section{K–Theory in Geometry and Topology}
2.1 Construction of the Grothendieck group\\
2.2 Definition and properties of K(X)\\
2.3 Functorial properties of K-Theory\\
2.4 Reduced K-Theory and suspension isomorphism
\section{K–Theory in Mathematical Physics}
3.1 Complex and real K-Theory\\
3.2 Bott periodicity theorem and its proof\\
3.3 Applications of Bott periodicity
\section{Introduction to Algebraic K–Theory}
4.1 Introduction to spectral sequences\\
4.2 Construction of the Atiyah-Hirzebruch spectral sequence\\
4.3 Applications in computing K-groups
\section{Higher Algebraic K–Theory and its Applications}
5.1 Definition and properties of Chern classes\\
5.2 The Chern character and its properties\\
5.3 Relationships between K-Theory and cohomology
\section{K–Theory in Algebraic Geometry and Number Theory}
6.1 The Thom isomorphism theorem\\
6.2 Gysin homomorphism and push-forward maps\\
6.3 Applications to cobordism theory
\section{K-Theory: Grothendieck Group and K(X) Definition}
7.1 Fredholm operators and analytical index\\
7.2 Topological index and K-homology\\
7.3 The Atiyah-Singer index theorem and its proof\\
7.4 Applications in differential geometry and topology
\section{Topological K-Theory: Bott Periodicity}
8.1 Definition and basic properties of equivariant K-Theory\\
8.2 Representation rings and character theory\\
8.3 Equivariant Bott periodicity and localization theorems
\section{The Atiyah–Hirzebruch Spectral Sequence}
9.1 Introduction to KK-Theory and its motivation\\
9.2 Basic constructions in KK-Theory\\
9.3 Applications to operator algebras and noncommutative geometry
\section{The Chern Character and Chern Classes}
10.1 K-Theory and characteristic classes\\
10.2 Applications to vector bundle classification\\
10.3 K-Theory and fixed point theorems\\
10.4 K-Theory in bordism and cobordism theory
\section{Thom Isomorphism: Theory and Applications}
11.1 K-Theory and D-branes in string theory\\
11.2 Topological insulators and K-Theory\\
11.3 K-Theory in quantum field theory
\section{Index Theory and the Atiyah–Singer Index Theorem}
12.1 K0 and K1 of rings\\
12.2 Quillen's higher algebraic K-Theory\\
12.3 Fundamental theorems in algebraic K-Theory
\section{Equivariant K–Theory and its Properties}
13.1 Milnor K-Theory and Bloch-Lichtenbaum spectral sequence\\
13.2 Motivic cohomology and algebraic K-Theory\\
13.3 Applications to algebraic cycles and motivic cohomology
\section{KK–Theory and its Applications}
14.1 K-Theory of schemes and varieties\\
14.2 Riemann-Roch theorems in algebraic geometry\\
14.3 K-Theory and zeta functions\\
14.4 Applications to arithmetic geometry

